\documentclass[12pt,a4paper]{article}
\usepackage[utf8]{inputenc}
\usepackage[french]{babel}
\usepackage{fancyheadings}
\usepackage{color}
\usepackage{graphicx}
\usepackage{epsfig}
\usepackage[top=1.9cm,hmargin=2.5cm]{geometry}
%\usepackage{fguill}
\usepackage{tipa}
\usepackage{longtable}
\newcommand{\gui}[1]{\og{#1}\fg}
\newcommand{\ouf}{\vspace{3mm}}
\usepackage{url}

\newcounter{serie} \setcounter{serie}1
\newcounter{numexo}
\newcommand{\exonum}{Q.\theserie.\addtocounter{numexo}{1}\thenumexo.}

%%%%%%%%%%%%%%%%%%%%%%%%%%%%%%%%%%%%%%%%%%%%%%%%%%%%%%%%%%%%%%%%%%%%%%%%%%%%%

\title{\vspace{-1em}Evaluation finale du compilateur\vspace{-2em}} 
%\author{}
\date{2018/2019}
%\date{4 janvier 2009}

\begin{document}
\setlength{\parindent}{0cm}

\maketitle

\lhead{\emph{Université Aix Marseille - L3 Informatique}}
\rhead{\emph{Compilation}}
%\setlength{\headrulewidth}{0.25pt} 
\thispagestyle{fancy}
%%%%%%%%%%%%%%%%%%%%%%%%%%%%%%%%%%%%%%%%%%%%%%%%%%%%%%%%%%%%%%%%%%%%%%%%%%%%%

\begin{longtable}{|l|p{5cm}|l|p{5cm}|} \hline
 NOM        & & PRÉNOM & \\ 
            & &        & \\ \hline
 NOM        & & PRÉNOM & \\ 
            & &        & \\ \hline 
 URL git~+  & \multicolumn{3}{|c|}{} \\
 no. commit & \multicolumn{3}{|c|}{} \\ \hline
\end{longtable}

\section{Le compilateur (4pt)}

\begin{longtable}{|p{2.7cm}|p{2.7cm}|p{2.7cm}|p{2.7cm}|p{2.7cm}|} \hline
lex & asynt & tab & 3a & nasm \\ \hline
& & & & \\ 
& & & & \\ \hline
\multicolumn{2}{|l|}{Extensions/améliorations} & \multicolumn{3}{c|}{} \\ 
\multicolumn{2}{|l|}{                        } & \multicolumn{3}{c|}{} \\ \hline
\multicolumn{2}{|l|}{Documentation/Readme    } & \multicolumn{3}{c|}{} \\ 
\multicolumn{2}{|l|}{                        } & \multicolumn{3}{c|}{} \\ \hline
\end{longtable}

\section{Exemples connus (4pt)}

\begin{longtable}{|p{2.2cm}|p{2.2cm}|p{2.2cm}|p{2.2cm}|p{2.2cm}|p{2.2cm}|} \hline
lex & asynt & tab & 3a & nasm & erreur\\ \hline
& & & & & \\ 
& & & & & \\ \hline
\end{longtable}

\section{Exemples nouveaux (4pt)}

\begin{longtable}{|p{2.2cm}|p{2.2cm}|p{2.2cm}|p{2.2cm}|p{2.2cm}|p{2.2cm}|} \hline
lex & asynt & tab & 3a & nasm & erreur\\ \hline
& & & & & \\ 
& & & & & \\ \hline
\end{longtable}

\section{Nouvelle fonctionnalité (8pt)}

\begin{longtable}{|p{2.2cm}|p{2.2cm}|p{2.2cm}|p{2.2cm}|p{2.2cm}|p{2.2cm}|} \hline
lex & asynt & tab & 3a & nasm & erreur\\ \hline
& & & & & \\ 
& & & & & \\ \hline
\end{longtable}


\end{document}
